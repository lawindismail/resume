\documentclass[11pt,a4paper]{article}

% For ATS-friendly PDF generation (searchable and copyable text)
\input{glyphtounicode}
\pdfgentounicode=1

% ---------- PACKAGES ----------
\usepackage[margin=20mm]{geometry}
\usepackage[T1]{fontenc}
\usepackage{lmodern} % A more modern alternative to the default Computer Modern font.
\usepackage[hidelinks]{hyperref} % For clickable links without distracting borders.
\usepackage{enumitem} % Advanced control over lists.
\usepackage{titlesec} % For customizing section titles.
\usepackage{array} % For more flexible table and array environments.
\usepackage{xcolor} % For using colors.
\usepackage{tabularx} % For tables with columns that automatically adjust their width.
\usepackage{ragged2e} % For better text justification.

% ---------- GLOBAL STYLES ----------
\setlength{\parindent}{0pt} % No indentation for paragraphs.
\setlength{\parskip}{0pt} % No extra space between paragraphs.

% Custom list style for tighter spacing in experience and project descriptions.
\setlist[itemize]{
    leftmargin=*,
    label=\textbullet,
    topsep=2pt,
    partopsep=0pt,
    itemsep=2pt,
    parsep=0pt
}

% Section title formatting for a clean, professional look.
\titleformat{\section}
  {\large\bfseries\scshape\RaggedRight} % Format: Large, bold, small caps, left-aligned.
  {}
  {0em}
  {}[\color{black}\titlerule] % A full-width line under the section title.
\titlespacing*{\section}{0pt}{10pt}{8pt} % Spacing: left, before, after.

\titleformat{\subsection}
  {\normalsize\bfseries\RaggedRight} % Format: Normal size, bold, left-aligned.
  {}
  {0em}
  {}
\titlespacing*{\subsection}{0pt}{8pt}{4pt} % Spacing: left, before, after.

\titleformat{\subsubsection}
  {\normalsize\bfseries\itshape\RaggedRight} % Format: Normal size, bold, italic, left-aligned.
  {}
  {0em}
  {}
\titlespacing*{\subsubsection}{0pt}{6pt}{3pt} % Spacing: left, before, after.

% ---------- DOCUMENT ----------
\begin{document}

% ===== HEADER =====
\begin{center}
    {\Huge \bfseries Lawind Ismail}
    \vspace{5pt}
\end{center}

\begin{center}
    \textit{Melbourne, VIC} \\
    \href{mailto:lawind.ismail@gmail.com}{lawind.ismail@gmail.com} \,|\,
    \href{https://www.linkedin.com/in/lawind}{linkedin.com/in/lawind} \,|\,
    \href{https://github.com/lawindismail}{github.com/lawindismail} \,|\,
    \textit{0477 119 068}
\end{center}

% ===== PROFESSIONAL SUMMARY =====
\section{Professional Summary}
\justifying
Hands-on mechatronics engineer with strong creativity, critical thinking, and a passion for problem-solving. I have worked on diverse set of projects such as programming micro-controllers to create custom communication protocols, integrating BMS into PCB's and developing fully functional mechanical assemblies in CAD. I am confident using a diverse set of engineering tools, including open-source alternatives. I thrive in collaborative settings, and a quick learner. Seeking roles in Melbourne, Australia; graduating at the end of 2025. Bilingual in English and Kurdish (Sorani).

% ===== SKILLS =====
\section{Skills}
{ % Group to keep the arraystretch change local to this table
\renewcommand{\arraystretch}{1.8} % Increase vertical spacing between rows
\begin{tabularx}{\textwidth}{@{} >{\bfseries}l X @{}}
    Software Development: & C, C++, Python, MATLAB, Simulink, AVR Assembly, Multithreaded Processing (PThreads), Git (GitHub), Linux, Visual Studio Code \\
    Embedded Systems \& IoT: & ESP-IDF, PlatformIO, Arduino, Microchip Studio / Atmel Studio, Micro-controllers (ESP32, Arduino Uno, ATMEGA32A), Custom Protocol Development, LoRa, Mesh Network Communication, Mobile Networks (4G/5G SIM), IIoT (Industrial IoT) \\
    Electronics \& PCB Design: & Circuit Design, PCB Design, Altium Designer, KiCAD, Antenna Design, Electrical Circuit Theory \\
    Mechanical \& Manufacturing: & SOLIDWORKS, Onshape, FreeCAD, Autodesk Inventor, Fusion 360, Technical Drawings (AS 1101), Sheet Metal Design, 3D Printing, Slicer Softwares (OrcaSlicer, Cura), Z-SUITE, CNC Machining (G-code/NC-code), DFMA/DFM/DFA, LEAN (Toyota Production System), Engineering Dynamics \& Statics, Materials Engineering, Materials Science \\
    Analysis, Simulation \& Controls: & Finite Element Analysis (FEA), Topology Optimisation (SOLIDWORKS \& nTop), ANSYS Granta Selector, Rockwell Arena, Control Systems, PLC (Programmable Logic Controllers), NI MyRIO, NI LabVIEW \\
    Project Management \& Tools: & Jira, Agile Methodology, Microsoft Project, Gantt Project, Microsoft Teams, Google Workplace, Microsoft Office 365 \\
    Professional Skills: & Microsoft Excel, Data Analytics, Engineering Mathematics, Linear Programming (Linear Optimization), Bill of Materials (BOM)  \\
    Soft Skills: & Communication, Teamwork, Active listening, Adaptability, Creativity, Reliable, Leadership, Time Management \\
\end{tabularx}
} % End the group

% ===== EDUCATION =====
\section{Education \& Qualifications}
\textbf{RMIT University} \hfill {2023--2025} \\
Bachelor of Engineering (Advanced Manufacturing \& Mechatronics) (Honours) \\
\textit{Currently enrolled; Weighted Average Mark (WAM): 80}
\newpage % This is because work experience is in the final line of page one
% ===== WORK EXPERIENCE =====
\section{Work Experience}
\textbf{Electronic, Hardware, Embedded, \& Mechanical Engineer} \textbar{} \textit{DataFarm (Startup)} \hfill
\vspace{-8pt} % Reduces the space above the date
\begin{flushright}
    {Nov 2023--Present}
\end{flushright}
\vspace{-8pt} % Reduces the space below the date
\textit{DataFarm streamlines agricultural data collection for farmers via compact soil sensors and a network that uploads real-time data to the cloud.}
\begin{itemize}
    \item Planned sprints and assigned tasks in Jira; maintained traceability across firmware, electronics, and mechanical workstreams.
    \item Developed ESP32 firmware in PlatformIO, implementing multithreaded processing and microcontroller sleep states to minimise power consumption.
    \item Built consumer-grade hardware in Onshape and SOLIDWORKS using DFMA principles; authored Engineering Drawings conforming to AS1101.
    \item Performed materials selection with ANSYS Granta Selector; produced comparative Ashby charts to justify down-selects.
    \item Designed schematics and PCBs in Altium Designer; created BOMs for economic feasibility analysis and supplier PCBA quotes.
    \item Managed version control using Git and GitHub; enforced branch strategy and code reviews across the team.
\end{itemize}

% ===== NOTABLE UNIVERSITY PROJECTS =====
\section{Notable University Projects}
\subsection{Capstone with Industry Partner: Sutton Tools Pty Ltd (In Progress)}
Automated feeder system with defect detection; responsibilities include system architecture, controls integration, and prototype validation.

\subsection{Robotic Gripper Project}
\begin{itemize}
    \item Designed and fabricated a fractal-inspired robotic gripper achieving 50 N gripping force while handling delicate objects (100\% success rate with eggs).
    \item Implemented closed-loop PID control in LabVIEW with $\pm 0.5$ mm positional accuracy; calibrated load cells to $\pm 0.1$ N and IR sensors to $\pm 2$ mm.
    \item Modelled 15 precision CAD components in Onshape; optimised geometry for strength and gentle manipulation.
    \item Conducted iterative FEA in ANSYS Workbench; added reinforcements in critical stress zones to prevent failure during gripping.
    \item Led a cross-functional team of 5 through an 8-week rapid prototyping cycle; delivered the prototype 2 weeks ahead of schedule.
\end{itemize}

\subsection{Advanced Manufacturing \& Design (Topology Optimisation)}
\begin{itemize}
    \item Redesigned a hydraulic manifold for additive manufacturing in Onshape, achieving 88.7\% volume reduction (364,868 mm\textsuperscript{3} $\rightarrow$ 41,245 mm\textsuperscript{3}) and 46 g final mass while preserving interfaces.
    \item Optimised internal channel network with curved, self-supporting pathways (overhangs $< 80^\circ$) to minimise pressure drops and eliminate supports; maintained 1.5 mm minimum wall thickness in high-stress zones.
    \item Applied DFAM: build orientation analysis and structural reinforcements enabled support-free FDM printing; benchmarked parameters (e.g., cooling and speed).
    \item Performed hand calculations for hole positioning, connectivity mapping, and volumetric analysis; produced detailed channel diameter/coordinate tables.
    \item Fabricated prototypes on an ELEGOO Neptune 3 Pro (eSun PLA, 215$^\circ$C, 60 mm/s); achieved 0.15 mm dimensional accuracy with \textasciitilde 4 h print time.
    \item Led CAD modelling and 3D printing in a 6-person team; owned final report, documentation, and design evaluation.
\end{itemize}

\subsection{Buck Converter PCB Design Project}
\begin{itemize}
    \item Engineered and simulated a 12 V $\rightarrow$ 5 V, 2 A step-down buck converter in Altium Designer; selected R/C values from first principles.
    \item Verified load-step and frequency response in simulation; achieved voltage ripple $< 50$ mV\textsubscript{pp} and efficiency $> 90$\% at full load.
    \item Produced a two-layer PCB with DFM rules: dedicated 1.5 mm$^2$ copper pour for power planes vs. 0.3 mm$^2$ signal traces; passed all DRCs.
    \item Assembled SMT components on a 100 $\times$ 100 mm PCB; minimised parasitic inductance in switching loops; validated regulation within $\pm 3$\%.
    \item Conformed to IPC standards: IPC-2221, IPC-2222, IPC-7351, IPC-4101; authored a technical report comparing simulation vs. experimental data and recommending an optimised output filter.
\end{itemize}

\subsection{Mechanical Evaluation of Titanium Metamaterials}
Executed cyclic compressive loading on Ti-6Al-4V lattice specimens (per Noronha et al., 2025); captured load--displacement at 50 Hz with synchronised high-speed video.
\begin{itemize}
    \item Characterised stress--strain over six cycles (5--40\% strain): toe-region settling, linear elasticity ($\le$ 10\%), ductile plasticity (15--25\%), delayed fracture initiation ($\sim 30$\%).
    \item Quantified damage accumulation via modulus degradation (16.93 GPa $\rightarrow$ 6.21 GPa) and yield strength progression (0.47 MPa $\rightarrow$ 0.86 MPa).
    \item Benchmarked against cubic and TPMS gyroid lattices: $\ge 52$\% increase in maximum strain capacity (35\% vs. $\le 23$\%) and shift from brittle collapse to ductile failure.
    \item Analysed energy absorption/compliance with spreadsheets and curve fitting; reported enhanced damage tolerance for aerospace, automotive, and biomedical use cases.
    \item Authored a standalone technical report covering methodology, results, and design implications.
\end{itemize}

% ===== REFERENCES =====
\section{References}
\begin{flushleft}
    \begin{tabular}{@{}lll@{}}
        Anas Koujan & Thermo Trade & 0426 823 552 \\
        Nicholas Alston & Gippsland Water & 0422 208 487 \\
        Dean Hooper & Siemens Energy & 0450 032 002 \\
        Mohnish Deshpande & Cisco Splunk & 0480 136 857 \\
    \end{tabular}
\end{flushleft}

\end{document}